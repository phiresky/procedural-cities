\documentclass{thesisclass}
% Based on thesisclass.cls of Timo Rohrberg, 2009
% ----------------------------------------------------------------
% Thesis - Main document
% ----------------------------------------------------------------

\usepackage{microtype}
\usepackage{amsmath}

$if(highlighting-macros)$
$highlighting-macros$
$endif$

$for(header-includes)$
$header-includes$
$endfor$

%% -------------------------------
%% |  Information for PDF file   |
%% -------------------------------
\hypersetup{
	breaklinks=true,
	bookmarks=true,
	pdfauthor={$author$},
	pdftitle={$title$},
	colorlinks=true,
	citecolor=$if(citecolor)$$citecolor$$else$blue$endif$,
	urlcolor=$if(urlcolor)$$urlcolor$$else$blue$endif$,
	linkcolor=$if(linkcolor)$$linkcolor$$else$magenta$endif$,
	pdfborder={0 0 0}
	$if(hidelinks)$,hidelinks,$endif$
}

\def\tightlist{}

%% ---------------------------------
%% | Information about the thesis  |
%% ---------------------------------

\newcommand{\myname}{$author$}
\newcommand{\mytitle}{$title$}
\newcommand{\myinstitute}{$institute$}

\newcommand{\reviewerone}{?}
\newcommand{\reviewertwo}{?}
\newcommand{\advisor}{?}
\newcommand{\advisortwo}{?}

\newcommand{\timestart}{XX. Monat 20XX}
\newcommand{\timeend}{XX. Monat 20XX}
\newcommand{\submissiontime}{DD. MM. 20XX}


%% ---------------------------------
%% | ToDo Marker - only for draft! |
%% ---------------------------------
% Remove this section for final version!
\setlength{\marginparwidth}{20mm}

\newcommand{\margtodo}
{\marginpar{\textbf{\textcolor{red}{ToDo}}}{}}

\newcommand{\todo}[1]
{{\textbf{\textcolor{red}{(\margtodo{}#1)}}}{}}


%% --------------------------------
%% | Old Marker - only for draft! |
%% --------------------------------
% Remove this section for final version!
\newenvironment{deprecated}
{\begin{color}{gray}}
{\end{color}}


%% --------------------------------
%% | Settings for word separation |
%% --------------------------------
% Help for separation:
% In german package the following hints are additionally available:
% "- = Additional separation
% "| = Suppress ligation and possible separation (e.g. Schaf"|fell)
% "~ = Hyphenation without separation (e.g. bergauf und "~ab)
% "= = Hyphenation with separation before and after
% "" = Separation without a hyphenation (e.g. und/""oder)

% Describe separation hints here:
\hyphenation{
% Pro-to-koll-in-stan-zen
% Ma-na-ge-ment  Netz-werk-ele-men-ten
% Netz-werk Netz-werk-re-ser-vie-rung
% Netz-werk-adap-ter Fein-ju-stier-ung
% Da-ten-strom-spe-zi-fi-ka-tion Pa-ket-rumpf
% Kon-troll-in-stanz
}

%%%%%%%%%%%%%%%%%%%%%%%%%%%%%%%%%
%% Here, main documents begins %%
%%%%%%%%%%%%%%%%%%%%%%%%%%%%%%%%%
\begin{document}

% titlepage
	% Remove the following line for German text
	%\selectlanguage{ngerman}

	\frontmatter
	\pagenumbering{roman}

	% coordinates for the bg shape on the titlepage
	\newcommand{\diameter}{20}
	\newcommand{\xone}{-15}
	\newcommand{\xtwo}{160}
	\newcommand{\yone}{15}
	\newcommand{\ytwo}{-253}

	\begin{titlepage}
	% bg shape
	\begin{tikzpicture}[overlay]
	\draw[color=gray]  
			 (\xone mm, \yone mm)
	  -- (\xtwo mm, \yone mm)
	 arc (90:0:\diameter pt) 
	  -- (\xtwo mm + \diameter pt , \ytwo mm) 
		-- (\xone mm + \diameter pt , \ytwo mm)
	 arc (270:180:\diameter pt)
		-- (\xone mm, \yone mm);
	\end{tikzpicture}
		\begin{textblock}{10}[0,0](4,2.5)
			\includegraphics[width=.3\textwidth]{logos/KITLogo_RGB.pdf}
		\end{textblock}
		\changefont{phv}{m}{n}	% helvetica	
		\vspace*{3.5cm}
		\begin{center}
			\Huge{\mytitle}
			\vspace*{2cm}\\
			\Large{
				Proseminar-Ausarbeitung von
			}\\
			\vspace*{1cm}
			\huge{\myname}\\
			\vspace*{1cm}
			\Large{
				An der Fakult\"at f\"ur Informatik
				\\
				\myinstitute
			}\\
			\vspace*{1cm}
			\Large{\today}
		\end{center}
		\vspace*{1cm}
	%\Large{
	%\begin{center}
	%\begin{tabular}[ht]{l c l}
	%  % Gutachter sind die Professoren, die die Arbeit bewerten. 
	%  \iflanguage{english}{Reviewer}{Erstgutachter}: & \hfill  & \reviewerone\\
	%  \iflanguage{english}{Second reviewer}{Zweitgutachter}: & \hfill  & \reviewertwo\\
	%  \iflanguage{english}{Advisor}{Betreuender Mitarbeiter}: & \hfill  & \advisor\\
	%  \iflanguage{english}{Second advisor}{Zweiter betreuender Mitarbeiter}: & \hfill  & \advisortwo\\
	%  % Der zweite betreuende Mitarbeiter kann weggelassen werden. 
	%\end{tabular}
	%\end{center}
	%}


	\vspace{2cm}


	\begin{textblock}{10}[0,0](4,16.8)
	\tiny{ 
			{KIT -- Universität des Landes Baden-Württemberg und nationales Forschungszentrum der Helmholtz-Gesellschaft}
	}
	\end{textblock}

	\begin{textblock}{10}[0,0](14,16.75)
	\large{
		\textbf{www.kit.edu} 
	}
	\end{textblock}

	\end{titlepage}
\blankpage


\tableofcontents
\blankpage


\mainmatter
\pagenumbering{arabic}

$body$

\Erklaerung
\end{document}
